\documentclass[final]{beamer}
\usetheme{RJH}
%~ \usepackage{ragged2e}
%~ \let\raggedright=\RaggedRight
\usepackage[orientation=landscape,size=a1,scale=1.4,debug]{beamerposter}
\usepackage[absolute,overlay]{textpos}
\usepackage[english]{babel}
\usepackage{color}
\setlength{\TPHorizModule}{1cm}
\setlength{\TPVertModule}{1cm}

\usepackage{lmodern}% http://ctan.org/pkg/lm

\title{Parallel Metropolis-Hastings Algorithm by Prefetching}
\author{Boyan Bejanov}
\footer{
\parbox{0.48\textwidth}{
    \flushleft \small
    COMP5704 Project, School of Computer Science, Carleton University
    }
\hfill
\parbox{0.48\textwidth}{\flushright
Bank of Canada}
}
\date{April 2014}

\begin{document}
\begin{frame}[t]{} 

%~ \vspace{-1ex}

\begin{columns}[t]
    \begin{column}{0.3\textwidth}

        \begin{block}{Introduction}
            \begin{description}
             \item[Metropolis-Hastings (MH)] is a Markov Chain Monte Carlo (MCMC)
                algorithm which is used to simulate random samples from
                probability distributions that are difficult to simulate
                otherwise.                
            \item[Prefetching] is a parallelization technique for the
                MH algorithm.  It is applicable in situations
                where the target p.d.f.,
                \textbf{ $\boldsymbol{\pi}\mathbf{(x)}$, is
                    \begin{itemize}
                        \item computationally very expensive;
                        \item impractical to parallelize.                
                    \end{itemize}
                }
            \end{description}
            E.g. Bayesian inference for  economic models
            %~ the posterior density of the model parameters is computed by Kalman
            %~ filter.
        \end{block}
       
        \begin{block}{MH algorithm}
            Markov chain with accept-reject transition rule:
            \begin{itemize}
                \item current state is $X_0$
                \item propose candidate $Y \sim q(X_0, \cdot)$
                \item accept the candidate with probability
                    $$\alpha = \frac{\pi(Y)q(Y,X_0)}{\pi(X_0)q(X_0,Y)}$$
            \end{itemize}
        \end{block}

        \begin{block}{Prefetching tree}
            \includegraphics[width=.9\textwidth]{PrefetchingTree.png}
            \begin{itemize}
            \item At each node the current state is in white
            \item The proposed candidate is in orange
            \end{itemize}
        \end{block}
     
    \end{column}
    \begin{column}{0.3\textwidth}

        \begin{block}{Implemetation}        
        \centering {
        \begin{tabular}[c]{|c|c||c||c|c||c|c|c|c||c|c|c|c|c|c|c|c|}
            \hline
            $k$ & 0 & 1 & 2 & 3 & 4 & 5 & 6 & 7 & 8 & 9 & 10 & 11 & 12 & 13 & 14 & 15\\
            \hline
            $s$ & 0 & 1 & \multicolumn{2}{c||}{2} & \multicolumn{4}{c||}{3} & \multicolumn{8}{c|}{4} \\
            \hline
            $c$ & $\ast$ & 0 & 0 & 1 & 0 & 1 & 2 & 3 & 0 & 1 & 2 & 3 & 4 & 5 & 6 & 7 \\
            \hline
        \end{tabular}
        }
        \begin{itemize}
            \item $X_k$ was proposed at step $s=\left\lfloor\log k\right\rfloor+1$
            \item $X_k$ was proposed from $X_c$, $c=k-2^{s-1}$
            \item The two children of proposal $X_k$ are $X_a$ and $X_r$ where
                $a = k+2^s$ and $r = a-2^{s-1}$
        \end{itemize}
        \end{block}

        \begin{block}{Full prefetching}
            \begin{itemize}
            \item Consider all $2^n$ possible paths $n$ steps ahead
            \item Compute $2^n-1$ evaluations of $\pi(\cdot)$ in parallel
            \item Run $n$ steps of the Markov chain sequentially 
            \end{itemize}
            
        \end{block}

        \begin{block}{Complexity of Full Prefetching}
            \begin{description}
                \item[Sequentail time] $\displaystyle T_s = n$ % \phantom{\frac{M}{M}}$
                \item[Parallel time] $\displaystyle T(n,p) = \frac{2^n-1}{p}$
                \item[Speedup] $\displaystyle s(n,p) = \frac{np}{2^n-1}$
                \item[N.B.] \textit{\textcolor{red}{Speedup  is not logarithmic in $p$}}
            \end{description}
        \end{block}

        \begin{block}{Speedup graph}
        \includegraphics[width=1\textwidth]{speedup-full.png}
        \end{block}
      
    \end{column}
    \begin{column}{0.3\textwidth}

       \begin{block}{Incomplete prefetching}
            \begin{itemize}
            \item Prefetch only $p$ proposals, not the full tree
            \item For $p<2^n-1$ the chain may not make $n$ steps
            \item By scaling the proposal distribution, $q(x,y)$, we can 
                control the acceptance rate, denoted $\alpha^*$
            \item At each branch the probability to accept is $\alpha^*$ and
                the probability to reject is $1-\alpha^*$
            \item The probability to reach given candidate is the product
                of all probabilities along its path
            \item Evaluate $p$ candidates that maximize the expected
                depth to be reached, denoted $D(p)$
            \end{itemize}
        \end{block}

        \begin{block}{Complexity of Incomplete Prefetching}
            \begin{description}
                \item[Parallel time] $\displaystyle T(p) = \frac{n}{D(p)}$
                \item[Speedup] $\displaystyle s(p) = D(p) \phantom{\frac{M}{M}}$
            \end{description}
            \vspace{2ex}
            \begin{itemize}
                \item E.g. for $\alpha^* = 0.234$
            \end{itemize}
            \centerline{
            \begin{tabular}{|c|c|c|c|c|c|c|c|c|c|c|} % c|c|c|c|c|}
                \hline
                $p$    &1  &2    &3     &4     &5     &6     &7     &8     &9   \\%  &10     &11    &12   &13   &14    & 15 \\
                \hline
                $D(p)$ & 1 &1.77 & 2.35 & 2.80 & 3.15 & 3.41 & 3.64 & 3.84 & 4.03 \\%& 4.20  & 4.36 &4.49 &4.63 & 4.77 & 4.88  \\
                \hline
            \end{tabular}
            }
        \end{block}

        \begin{block}{Speedup graph}
        \includegraphics[width=1\textwidth]{speedup-partial.png}
        \end{block}
      
    \end{column}
\end{columns}

\end{frame}
\end{document}
